\section{The Language of Mathematics}
\hrulefill
\subsection{Variables}

\paragraph*{Definition} A \textbf{variable} is a symbol that is used as a placeholder when:
\begin{itemize}
    \item The quantity has one of more values, but is not known.
    \begin{itemize}
        \item For example: $2x^2 - x = 7$
    \end{itemize}
    \item The quantity represents \textbf{any element} from a given set.
    \begin{itemize}
        \item For example: The reciporical of any non-zero integer $n$ is $\frac{1}{n}$.
    \end{itemize}
\end{itemize}

\paragraph*{Writing Sentences using Variables}
We can rewrite the following sentences using variables:
\begin{itemize}
    \item Is there an integer $n$ that has a remainder of 2 when it is divided by 5?
    \begin{itemize}
        \item Is there an integer $n$ such that $n \% 5 = 2$?
    \end{itemize}
    \item The cube root of any negative real number is negative.
    \begin{itemize}
        \item For any real number $s$, if $s < 0$, then $\sqrt[3]{s} < 0$.
    \end{itemize}
\end{itemize}

\paragraph*{Types of Statements}
\begin{itemize}
    \item A \textbf{universal statement} is a statement that is true always true.
    \begin{itemize}
        \item For example: \textbf{All} positive numbers are greater than 0.
    \end{itemize}

    \item A \textbf{conditional statement} is a statement that is true if a certain condition is met.
    \begin{itemize}
        \item For example: \textbf{If} 378 is divisible by 18, \textbf{then} 378 is divisible by 6.
    \end{itemize}

    \item A \textbf{universal conditional statement} is a statement that is both conditional and universal.
    \begin{itemize}
        \item For example: \textbf{For all} animals $a$, if $a$ is a dog, \textbf{then} a is a mammal.
        \item As a universal statement: \textbf{For all} dogs $a$, $a$ is a mammal.
        \item As a conditional statement: \textbf{If} $a$ is a dog, \textbf{then} $a$ is a mammal.
    \end{itemize}

    \item An \textbf{existential statement} gives a property that is true for at least one thing.
    \begin{itemize}
        \item \textbf{There is} a prime number that is even.
    \end{itemize}

    \item A \textbf{universal existential statement} is a statement where the first part is universal and the second part is existential.
    \begin{itemize}
        \item \textbf{Every} real number \textbf{has} an additive inverse.
        \item \textbf{For all} real numbers $r$, \textbf{there is} an additive inverse $-r$.
        \item \textbf{For all} real numbers $r$, \textbf{there is} a real number $s$ such that $r + s = 0$.
    \end{itemize}

    \item An \textbf{existential universal statement} is a statement where the first part is existential and the second part is universal.
    \begin{itemize}
        \item \textbf{There is} a positive integer that is less than or equal to \textbf{every} positive integer.
        \item \textbf{There is} a positive intefer $m$ such that \textbf{every} positive integer is greater than or equal to $m$.
        \item \textbf{There is} a positive integer $m$ with the property that \textbf{for all} positive integers $n$, $m \leq n$.
    \end{itemize}
\end{itemize}