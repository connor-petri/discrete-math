\subsection{Conditional Statements}
\hrulefill
\paragraph*{Definition}
A Conditional statement is in the form "If $p$, then $q$" and is denoted as $p \implies q$ This is read as $p$ implies $q$.\\
\begin{itemize}
    \item $p$ is the \textbf{hypothesis} of the statement.
    \item $q$ is the \textbf{conclusion} of the statement.
\end{itemize}

\paragraph*{Order of Operations}
\begin{itemize}
    \item (): parentheses
    \item $\neg$: negation
    \item $\land/\lor$: conjunction/disjunction
    \item $\implies$: implication
\end{itemize}

\paragraph*{Equivalent of Conditional Statements}
\begin{align*}
    p \implies q &\equiv \neg p \lor q\\
    \neg (p \implies q) &\equiv p \land \neg q\\
\end{align*}

\paragraph*{Example}
Find the negation of the following statement: "If my car is in the repair shop then I cannot go to class".
\begin{itemize}
    \item Hypothesis ($p$): "My car is in the repair shop"
    \item Conclusion ($q$): "I cannot go to class"
    \item Convert: $p \implies q \equiv \neg p \lor q$
    \item Negation: $\neg (p \implies q) \equiv \neg (\neg p \lor q) \equiv p \land \neg q$
    \item Convert back: "My car is in the repair shop and I can go to class"
\end{itemize}

\paragraph*{Negation vs Inverse}
The negation of a statement is NOT the same as the inverse of the statement.
\begin{itemize}
    \item Negation: $\neg (p \implies q)$
    \item Inverse: $\neg p \implies \neg q$
\end{itemize}

\paragraph*{Example}
If p is a square, then p is a rectangle.
\begin{itemize}
    \item Hypothesis ($p$): "p is a square"
    \item Conclusion ($q$): "p is a rectangle"
    \item Negation: $\neg (p \implies q) \equiv p \land \neg q$
    \item Convert back: "p is a square and p is not a rectangle"
    \item Inverse: $\neg p \implies \neg q \equiv p \lor \neg q$
    \item Convert: "If p is not a square, then p is not a rectangle"
\end{itemize}

\paragraph*{More statement types}
\begin{itemize}
    \item Contrapositive of $p \implies q \equiv \neg q \implies \neg p$
    \item Converse of $p \implies q \equiv q \implies p$
    \item Inverse of $p \implies q \equiv \neg p \implies \neg q$
\end{itemize}

\paragraph*{Example}
If today is Easter then tomorrow is Monday.
\begin{itemize}
    \item Hypothesis ($p$): "Today is Easter"
    \item Conclusion ($q$): "Tomorrow is Monday"
    \item Convert: $p \implies q$
    \item Contrapositive: $\neg q \implies \neg p \equiv \text{If tomorrow is not Monday, then today is not Easter}$
    \item Converse: $q \implies p \equiv \text{If tomorrow is Monday, then today is Easter}$
    \item Inverse: $\neg p \implies \neg q \equiv \text{If today is not Easter, then tomorrow is not Monday}$
\end{itemize}

\paragraph{Biconditional Statements}
A biconditional statement is in the form "p if and only if q" and is denoted as $p \iff q$. This is read as $p$ if and only if $q$.\\
\begin{equation}
    p \iff q \equiv (p \implies q )\land (q \implies p)
\end{equation}

\paragraph*{Sufficient and Necessary Conditions}
If $r$ and $s$ are statements:
\begin{itemize}
    \item r is a \textbf{sufficient condition} for s if $r \implies s$.
    \item r is a \textbf{necessary condition} for s if $s \implies r$ or $s \implies r$.
    \item r is a \textbf{necessary and sufficient condition} for s if $r \iff s$.
\end{itemize}